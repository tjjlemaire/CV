%!TEX program = xelatex
\documentclass[a4paper]{cv}
\englishfalse
\begin{document}
%----------------------------------------------------------------------------------------
%	TITLE SECTION
%----------------------------------------------------------------------------------------
\hspaceleft\begin{minipage}{0.85\textwidth}
\name{Théo Lemaire}\\
\tag{Ingénieur \tbl{} Neuroscientifique \tbl{} Programmeur versatile}\\
\details
\end{minipage}
\begin{minipage}{0.15\textwidth}
\photo
\end{minipage}

%----------------------------------------------------------------------------------------
%	LEFT COLUMN
%----------------------------------------------------------------------------------------
\hspaceleft\begin{minipage}[t]{0.6\textwidth}

\experience

\runsubsection{Doctorat en Neurosciences computationnelles}
\descript{| \href{http://www.http://tne.epfl.ch/}{TNE Lab, EPFL}}
\timeplace{Depuis Avr 2016}{Campus Biotech, Geneva, CH}
Dvpt. de modèles computationnels pour comprendre et optimiser la \emph{Neuromodulation par Ultrasons} aux échelles cellulaire et anatomique.
\sectionspace

\runsubsection{Moniteur de Ski Alpin}
\descript{| \href{http://www.esf.net}{Ecole du Ski Français}}
\timeplace{Depuis Janv 2013}{Monts Jura, FR}
Leçons privées et collectives aux skieurs de tous âges et tous niveaux. En charge d'un groupe compétition pendant 4 ans. Formation au diplôme d’état en cours.
\sectionspace

\runsubsection{Professeur de Mathématiques}
\descript{| \href{http://www.jda-gex.org/college}{Institution Jeanne d'Arc}}
\timeplace{Nov 2015 - Déc 2015}{Gex, FR}
Enseignement à 3 classes de collège (environ 75 élèves, entre 10 et 15 ans).
\sectionspace

\runsubsection{Stagiaire Ingénieur Software}
\descript{| \href{http://www.zenithtechnologies.com/}{Zenith Technologies}}
\timeplace{Avr – Août 2013}{Cork, IRL}
Dvpt. d'un programme \emph{C++} extrayant des informations d'une base de données pour fournir aux chefs d’équipe un aperçu global de l'évolution de leur projet.
\sectionspace

\runsubsection{Stagiaire en Cinésiologie}
\descript{| \href{http://www.hug-ge.ch/chirurgie-orthopedique-traumatologie-appareil/laboratoire-cinesiologie}{Hôpitaux Universitaires de Genève}}
\timeplace{Août 2012 - Janv 2013}{Genève, CH}
Dvpt. d'une application \emph{Matlab} pour analyser les données biomécaniques d’examens cliniques, utilisée pour les rapports et publications scientifiques.
\sectionspace

\projects

\runsubsection{Projet de Master en Neuroprothèses}
\descript{| \href{http://tne.epfl.ch/}{TNE Lab, EPFL}}
Dvpt. de modèles computationnels afin de prédire les performances de différents types d'électrode de stimulation implantées dans un nerf périphérique.
\sectionspace

\runsubsection{Projet de Biorobotique}
\descript{| \href{http://dhlab.epfl.ch/}{DH Lab, EPFL}}
Dvpt. d'un outil de vision par ordinateur et d'une stratégie de navigation permettant à un robot d'évoluer en slalom à travers des portes rectangulaires.
\sectionspace

\runsubsection{Projet en Humanités Digitales}
\descript{DH Lab, EPFL}
Dvpt. d'un nouveau modèle épidémique spatio-temporel pour étudier la propagation de la peste à Venise au moyen-âge. \weblink{http://veniceatlas.epfl.ch/atlas/experience/simulation/the-plague/}{Venice Atlas}
\sectionspace

\techskills

\def\arraystretch{1.5}
\begin{tabular}{R{0.21\textwidth} L{0.7\textwidth}}

\cvskill{python.pdf}{Python} & Outils de calcul \& analyse (\emph{numpy} - \emph{scipy} - \emph{pandas} - \emph{matplotlib}) \tbl{} Machine learning (\emph{scikit-learn}) \tbl{} Systèmes EDP \& modèles FEM \tbl{} Multi-threading/processing \tbl{} Simulations \href{https://neuron.yale.edu/neuron/}{\emph{NEURON}} \tbl{} \emph{Jupyter notebooks} \tbl{} Tâches d'automatisation \tbl{}Interaction avec APIs\\

\cvskill{gears.pdf}{\CC} & Programmation orientée objet \tbl{} Flux IO \tbl{} Requêtes XML \tbl{} GUIs \tbl{} Multi-threading (\href{https://www.boost.org/}{\emph{Boost}}) \tbl{} Graphiques 3D (\href{https://www.opengl.org/}{\emph{OpenGL}}) \tbl{} librairies mathématiques (\href{http://fftw.org/}{FFTW} , \href{http://eigen.tuxfamily.org/index.php?title=Main_Page}{Eigen})\\

\cvskill{matlab.pdf}{Matlab} & Calcul scientifique \tbl{} Machine learning \tbl{} UIs \tbl{} Requêtes SQL\\

\cvskill{frontend.pdf}{Front-end web} & Pages web adaptives (\emph{Javascript} - \emph{HTML} - \emph{CSS} - \href{http://getbootstrap.com/}{\emph{Bootstrap}}) \tbl{} Visualisations interactives (\href{https://d3js.org}{\emph{D3JS}} - \href{https://plot.ly/}{\emph{Plotly}}) \tbl{} Composants UI interactifs (\href{https://reactjs.org}{\emph{React.js}} - \href{https://dash.plot.ly/}{\emph{Dash}}) \\

\cvskill{msword.pdf}{MS Office} & Word - Excel - Powerpoint \tbl{} Automatisation avec Python / VBA\\

\end{tabular}

\vspace{5pt}
\otherskills

\end{minipage}
\hsepcol
%----------------------------------------------------------------------------------------
%	RIGHT COLUMN
%----------------------------------------------------------------------------------------
\begin{minipage}[t]{0.33\textwidth}

\education

\coursework

\subsection{Graduate}
Neuroprosthèses sensorimotrices\\
Bioelectronique flexible\\
Traitement d'image \tbl{} Machine learning\\
Systèmes dynamiques \tbl{} Bioméchanique\\
Analyse \& modélisation de la marche\\
Control moteur computationel\\
Bioinformatique \tbl{} Biologie des systèmes\\
Humanités Digitales
\sectionspace

\subsection{Undergraduate}
Analyse \tbl{} Algèbre \tbl{} Physique\\
Chimie \tbl{} Chimie organique\\
Biologie cellulaire \tbl{} Biologie moléculaire\\
Analyse numérique \tbl{} Statistiques\\
Electronique \tbl{} Traitement du signal\\
Programmation (C | C++ | Matlab)\\
Biologie du dvpt \tbl{} Microbiologie\\
Physiologie \tbl{} Génétique \tbl{} Génomique\\
Dynamique des fluides \tbl{} Transport\\
Biothermodynamique \tbl{} Neuroscience
\sectionspace

\languages

\hobbies

\end{minipage}
%----------------------------------------------------------------------------------------
%	VERSO
%----------------------------------------------------------------------------------------
\publications
\end{document}