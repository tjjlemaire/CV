%!TEX program = xelatex
\documentclass[a4paper]{cv}
\begin{document}
\pagecolor{bg}
%----------------------------------------------------------------------------------------
%	TITLE SECTION
%----------------------------------------------------------------------------------------

\begin{center}
\name{Théo Lemaire}\\
\vspace{5pt}
\tag{Bioingénieur \& programmeur versatile}\\
\vspace{-5pt}
\address{Rue des Maraîchers 46}{1205 Genève}{CH}\ \ \ \email{theo.lemaire1@gmail.com}\ \ \ \mobilephone{+41 79 629 39 05}\ \ \ \linkedin{https://ch.linkedin.com/in/theolemaire}{theolemaire}
\end{center}

%----------------------------------------------------------------------------------------
%	LEFT COLUMN
%----------------------------------------------------------------------------------------
\hspaceleft\begin{minipage}[t]{0.6\textwidth}

\section{\texorpdfstring{\faBriefcase}\ \ Expérience}\sectionline

\runsubsection{Doctorat en Neuroingénierie}
\descript{| \href{http://www.http://tne.epfl.ch/}{TNE Lab, Campus Biotech}}
\timeplace{Depuis Avr 2016}{Geneva, CH}
Développement d'un cadre de modélisation multiscalaire pour comprendre et optimiser la neuromodulation par ultrasons.
\sectionspace

\runsubsection{Moniteur de Ski Alpin}
\descript{| \href{http://www.esf.net}{Ecole du Ski Français}}
\timeplace{Depuis Janv 2013}{Monts Jura, FR}
5 saisons d'enseigement à tous les publics: privés, collectifs et scolaires, enfants et adultes, débutants à compétiteurs. Formation au diplôme d’état en cours.
\sectionspace

\runsubsection{Professeur de Mathématiques}
\descript{| \href{http://www.jda-gex.org/college}{Institution Jeanne d'Arc}}
\timeplace{Nov 2015 - Déc 2015}{Gex, FR}
Enseignement des mathématiques à 3 classes de collège (6e, 5e, 3e) durant 1 mois. Gestion de l'autorité et instauration d'une dynamique d'apprentissage.
\sectionspace

\runsubsection{Stagiaire Ingénieur Software}
\descript{| \href{http://www.zenithtechnologies.com/}{Zenith Technologies}}
\timeplace{Avr – Août 2013}{Cork, IRL}
Création d'une application C++ pour extraire des données d'une base de données \href{http://www3.emersonprocess.com/deltav/version13/}{\emph{DeltaV}} et fournir aux chefs d’équipe un aperçu global de l'évolution de leur projet. Design de scripts VBA utilisés au quotidien pour générer de la documentation.\sectionspace

\runsubsection{Stagiaire en Cinésiologie}
\descript{| \href{http://www.hug-ge.ch/chirurgie-orthopedique-traumatologie-appareil/laboratoire-cinesiologie}{Hôpitaux Universitaires de Genève}}
\timeplace{Août 2012 - Janv 2013}{Genève, CH}
Création d'une application Matlab (interface, outils de traitement \& de traçage, base de données, génération de PDF) pour analyser les données biomécaniques d’examens cliniques. Utilisé pour les rapports et publications scientifiques.
\sectionspace

\section{\texorpdfstring{\faLineChart} \ \ Projets académiques}\sectionline

\runsubsection{Projet de Master en Neuroprothèses}
\descript{TNE Lab, EPFL}
Modélisation numérique d’interfaces nerf-électrode avec la plateforme  \href{http://www.zurichmedtech.com/sim4life/}{\emph{Sim4Life}} pour améliorer le développement de neuroprothèses du membre supérieur.
\sectionspace

\runsubsection{Projet de Biorobotique}
\descript{BIOROB Lab, EPFL}
Développement d’un outil de vision par ordinateur sur le simulateur \href{https://www.cyberbotics.com/overview}{\emph{Webots}} pour la détection de portes colorées et la navigation d'un robot à roues différentielles.
\sectionspace

\runsubsection{Projet en Humanités Digitales}
\descript{DH Lab, EPFL}
Simulation numérique d'un nouveau modèle épidémique spatio-temporel pour étudier la propagation de la peste à Venise au moyen-âge. \weblink{http://veniceatlas.epfl.ch/atlas/experience/simulation/the-plague/}{Venice Atlas}
\sectionspace

\section{\texorpdfstring{\faWrench}\ \ Compétences techniques}\sectionline

% \mlovalbox{\textbf{C++} (\approx \emph{3000 hrs}): POO \tbl{} Flots IO \tbl{} Parsing \& sérialisation \tbl{} Requêtes XML \tbl{} Multithreading av. \href{https://www.boost.org/}\emph{Boost} \tbl{} Interfaces \href{https://www.wxwidgets.org/}{\emph{wxWidgets}} \tbl{} Graphiques 3D av. \href{https://www.opengl.org/}{\emph{OpenGL}}}
% \mlovalbox{\textbf{MATLAB} (\approx \emph{2000 hrs}): Calcul scientifique \tbl{} Traitement signal numérique \tbl{} Machine learning \tbl{} Interaction av. base de données SQL \tbl{} Interfaces graphiques}
% \mlovalbox{\textbf{Python} (\approx \emph{3000 hrs}): Calcul scientifique (\emph{NumPy}, \emph{SciPy}) \tbl{} Intégration de systèmes d'EDO \tbl{} Interfaces av. \emph{Tkinter} \tbl{} Animations 3D av. \emph{VPython} \tbl{} Communication av. Excel \tbl{} Emailing \tbl{} Interaction API \tbl{} Documentation av.  \emph{Sphinx}}
% \mlovalbox{\textbf{Front-End Web} (\approx \emph{200 hrs}): Pages web adaptives av. \emph{Javascript}, \emph{HTML5}, \emph{CSS3} \& \href{http://getbootstrap.com/}{\emph{Bootstrap}} \tbl{} Graphiques \& animations av. \href{https://d3js.org}{\emph{D3JS}}}
% \ovalbox{\textbf{MS Office \& VBA}} \ \ovalbox{\textbf{LaTeX}} \ \ovalbox{\textbf{Illustrator}} \ \ovalbox{\textbf{LabVIEW}} \ \ovalbox{\textbf{Model. Eléments Finis}}


\wheelchart{1.5cm}{0.5cm}{%
  6/12em/blue3/{\cvskill{matlab.pdf}{Matlab}\\ Calcul scientifique\\ Machine learning \tbl{} Interfaces \\ Interaction av. base de données},
  6/12em/blue2/{\cvskill{python.pdf}{Python}\\ Calcul scientifique\\ Interfaces \tbl{} Animations 3D \\ Model. Eléments Finis\\ NEURON},
  2/12em/blue4/{\cvskill{vba.pdf}{Visual Basic}\\ Macros pour Word \& Excel},
  6/12em/blue1/{\cvskill{gears.pdf}{\CC}\\ Prog. orientée object\\ Multithreading\\ Interfaces \tbl{} Graphiques 3D},
  3/12em/blue5/{\cvskill{frontend.pdf}{Front-End}\\ \emph{Javascript} \tbl{} \emph{HTML5} \tbl{} \emph{CSS3}\\ \emph{Bootstrap} \tbl{} \emph{D3JS}}
}

\vspace{10pt}
\centering{\cvskill{msword.pdf}{MS Office} \tbl{} \cvskill{blank.pdf}{\hspace{-5pt}\LaTeX} \tbl{} \cvskill{illustrator.pdf}{Illustrator} \tbl{} \cvskill{labview.pdf}{LabVIEW} \tbl{} \cvskill{git.pdf}{Git}}


\end{minipage}
\hsepcol
%----------------------------------------------------------------------------------------
%	RIGHT COLUMN
%----------------------------------------------------------------------------------------
\begin{minipage}[t]{0.33\textwidth}

\section{\texorpdfstring{\faGraduationCap} \ \ Formation}\sectionline

\subsection{Master en Bioingénierie avec Mineur en Neuroprothèses}
\descript{\href{http://www.epfl.ch}{EPF Lausanne}}
\timeplace{Sept 2013 - Sept 2015}{Lausanne, CH}
Moyenne: 5.34 / 6.0
\sectionspace

\subsection{Bachelor en Sciences \& Technologies du Vivant}
\descript{\href{http://www.epfl.ch}{EPF Lausanne}}
\timeplace{Sept 2009 - Juil 2012}{Lausanne, CH}
Moyenne: 4.92 / 6.0
\sectionspace

\subsection{Baccalauréat scientifique}
\descript{Lycée Int. Ferney Voltaire}
\timeplace{Sept 2006 - Juil 2009}{Ferney, FR}
moyenne: 18.71 / 20.0
\sectionspace

\section{\texorpdfstring{\faBook} \ \ Cours Suivis}\sectionline

\subsection{Graduate}
Neuroprosthèses sensorimotrices\\
Bioelectronique flexible\\
Traitement d'image \tbl{} Machine learning\\
Systèmes dynamiques \tbl{} Bioméchanique\\
Analyse \& modélisation de la marche\\
Control moteur computationel\\
Bioinformatique \tbl{} Biologie des systèmes\\
Humanités Digitales
\sectionspace

\subsection{Undergraduate}
Analyse \tbl{} Algèbre \tbl{} Physique\\
Chimie \tbl{} Chimie organique\\
Biologie cellulaire \tbl{} Biologie moléculaire\\
Analyse numérique \tbl{} Statistiques\\
Electronique \tbl{} Traitement du signal\\
Programmation (C | C++ | Matlab)\\
Biologie du dvpt \tbl{} Microbiologie\\
Physiologie \tbl{} Génétique \tbl{} Génomique\\
Dynamique des fluides \tbl{} Transport\\
Biothermodynamique \tbl{} Neuroscience
\sectionspace

\section{\texorpdfstring{\faComments} \ \ Langues}\sectionline

\noindent\begin{tabular}{@{}ll}
\textbf{Français} & \fivestars \\
\textbf{Anglais} & \fourstarshalf \\
\textbf{Allemand} & \threestarshalf \\
\textbf{Russe} & \onestar \\
\end{tabular}
\sectionspace

\section{\texorpdfstring{\faThumbsUp}\ \ Hobbies}\sectionline

\noindent\begin{tabular}{@{}c@{}c@{}c@{}c@{}}
\glyph{atom.pdf}{Science} & \glyph{taekwondo.pdf}{Taekwondo} & \glyph{football.pdf}{Football} & \glyph{television.pdf}{Séries TV}\\
\glyph{skiing.pdf}{Ski} & \glyph{mountains.pdf}{Randonnée} & \glyph{cycling.pdf}{Vélo} & \glyph{travel.pdf}{Voyages}\\
\end{tabular}

\end{minipage}
\end{document}